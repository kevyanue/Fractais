\chapter{Fractais gerados por IFS}
\section{Transformações lineares}
\subsection{Definição}

Seja \(U\) espaço vetorial sobre um corpo \( \mathbb{K} \), então
\[T:U \longrightarrow U\]
\\
é uma transformação linear se:

\begin{equation}
    T(\Vec{u} + \Vec{v}) = T(\Vec{u}) + T(\Vec{v})\ , \ \Vec{u}, \Vec{v} \in U
\end{equation}

\begin{equation}
    T(\alpha\Vec{u}) = \alpha T(\Vec{u}) \ , \ \alpha \in \mathbb{K}
\end{equation}

\begin{theorem}
se \(T\) é linear então \( T(0) = 0 \)
\end{theorem}

\begin{remark}
Neste capitulo trabalharemos em \( \mathbb{R}^{n} \), frequentemente em \( \mathbb{R}^{2} \) com a base canônica. T será sempre dado por:
\[ T: \mathbb{R}^n \longrightarrow \mathbb{R}^n \]
para algum \(n\)
\end{remark}

\subsection{Exemplos}

Tomemos T definido por

\[ T: \mathbb{R}^2 \longrightarrow \mathbb{R}^2 \]
\[ T(x, y) = (x, -y) =
  \left[ 
  {
    \begin{array}{c}
    x \\
   -y \\
    \end{array} 
    } 
  \right]
\]
note que aplicando as transformações nas bases canônicas obtemos
\[
T(1,0) = 
(1, 0) =
\left[ 
    {
        \begin{array}{c}
            1      \\
            0      \\
        \end{array} 
    } 
\right]
\]
\[
T(0,1) = 
(0, -1) =
\left[ 
    {
        \begin{array}{c}
            0      \\
            -1      \\
        \end{array} 
    } 
\right]
\]
E que 
\[ T(x, y) =
    \left[ 
    {
        \begin{array}{c c}
            1 &  0 \\
            0 & -1 \\
        \end{array} 
    } 
    \right]
    \left[ 
    {
        \begin{array}{c}
            x \\
            y \\
        \end{array} 
    } 
    \right]
    =
    \left[ 
    {
        \begin{array}{c}
            x \\
            -y \\
        \end{array} 
    } 
    \right]
\]
ou seja, \textbf{as imagens das transformações na base canônica de \( \mathbb{R}^{2} \) são precisamente as colunas da matriz que multiplica (x, y)}

\begin{remark}

mais genericamente, podemos descrever uma transformação \(T\) qualquer como


\[
    T: \mathbb{R}^{n} \longrightarrow \mathbb{R}^{n} 
\]
\[
    T(x) = A_{n \times n}x_{n \times 1} 
\]
Onde \(x\) é vetor do \(\mathbb{R}^{n}\) e \( A_{n \times n}\) é uma matriz \(n\) por \(n\).

Podemos definir \(A_{n x n}\) a partir das imagens de \( (1,0) \) e \( (0,1) \) como vetores coluna:
\[
    \left[ 
        {
            \begin{array}{c c}
                |\,    & |\,      \\
                T(1,0) & T(0,1)   \\
                |\,    & |\,      \\
            \end{array} 
        } 
    \right]
\]
\end{remark}

\section{Contração e Expansão}
\subsection{Definição}
\begin{definition}[contração]
    em geral, definimos que uma transformação \( T \) é uma contração quando:
        \[ 
        \exists \alpha \in [ \, 0,1\,) \, \, tal\, que
        \]
        \[
        \parallel T(u) - T(v) \parallel \: \leq \alpha \parallel u -v \parallel
        \]
    E definimos uma expansão de forma análoga, apenas invertendo o sinal da desigualdade.
    
\end{definition}

\subsection{exemplos}

\begin{proposition}\label{contract}
    Dado uma transformação T:
    \[
        T: \mathbb{R}^2 \longrightarrow \mathbb{R}^2
    \]
    \begin{equation}\label{escalar}
        T(x, y) = 
          \alpha
          \left[ 
            {
                \begin{array}{c c}
                    1 & 0 \\
                    0 & 1 \\
                \end{array} 
            } 
          \right]
          \left[ 
            {
                \begin{array}{c}
                    x \\
                    y \\
                \end{array} 
                } 
           \right]
           \, , \: \alpha \in \mathbb{R}
    \end{equation}
    se \( \alpha \in (0,1)\), então T é uma contração.
\end{proposition}
Neste caso, \(\alpha\) é o \textbf{Fator de contração} dessa transformação e ele que é usado no calculo da dimensão.
pode acontecer de queremos contrair os eixos de maneiras diferentes, como em
\[
E(x, y) = 
  \left[ 
    {
        \begin{array}{c c}
            \beta & 0      \\
            0     & \delta \\
        \end{array} 
    } 
  \right]
  \left[ 
    {
        \begin{array}{c}
            x \\
            y \\
        \end{array} 
        } 
   \right]
   ,
   \quad \beta,\delta \in [\,0, 1\,)
\]
Mas podemos visualizar \(E\) como:
\[
    E(x,y) = (\beta x, \delta y)
\]
Neste caso, o fator de contração é o \(min\{\beta, \delta\}\)
\section{Rotação}
\subsection{Definição}
seja

\[
    T: \mathbb{R}^2 \longrightarrow \mathbb{R}^2
\]
\[
    T(x) = R_{n \times n}x_{n \times 1}
\]
Usando o mesmo método de antes, podemos definir \(R\) a partir de suas imagens na base ortonormal

\begin{gather*}
    R(1,0) = R(\cos{\theta}, \sin{\theta})\\
    R(0,1) = R( - \sin{\theta}, \cos{\theta})
\end{gather*}
Então temos que
\begin{equation}
    R(x, y) = 
    \left[ 
    {
        \begin{array}{c c}
            \cos{\theta} & -\sin{\theta} \\
            \sin{\theta} &  \cos{\theta} \\
        \end{array} 
    } 
    \right]
    \left[ 
    {
        \begin{array}{c}
            x \\
            y \\
        \end{array} 
        } 
    \right]
\end{equation}

\subsection{exemplos}

\section{Transformações afins}

\subsection{Translação}
Seja
\[
    T: \mathbb{R}^2 \longrightarrow \mathbb{R}^2
\]
\begin{equation}
    T(x, y) = 
    \left[ 
    {
        \begin{array}{c}
            x \\
            y \\
        \end{array} 
        } 
    \right]
    +
    \left[ 
    {
        \begin{array}{c}
            a_1 \\ 
            a_2 \\
        \end{array} 
    } 
    \right]
    ,\quad a_1, a_2 \in \mathbb{R}
\end{equation}

\subsection{definição}

\begin{definition}

Uma \textbf{transformação afim} é uma transformação linear que admite translação e tem a forma:

\end{definition}

\begin{equation}
    A(x, y) = 
    \left[ 
    {
        \begin{array}{c c}
            \cos{\theta} & -\sin{\theta} \\
            \sin{\theta} &  \cos{\theta} \\
        \end{array} 
    } 
    \right]
    \left[ 
    {
        \begin{array}{c}
            \alpha  x \\
            \beta   y \\
        \end{array} 
        } 
    \right]
    +
    \left[ 
    {
        \begin{array}{c}
            a_1 \\ 
            a_2 \\
        \end{array} 
    } 
    \right]
\end{equation}
e se \( \beta = \delta\)

\begin{equation}
    A(x, y) =
    \alpha
    \left[ 
    {
        \begin{array}{c c}
            \cos{\theta} & -\sin{\theta} \\
            \sin{\theta} &  \cos{\theta} \\
        \end{array} 
    } 
    \right]
    \left[ 
    {
        \begin{array}{c}
            x \\
            y \\
        \end{array} 
        } 
    \right]
    +
    \left[ 
    {
        \begin{array}{c}
            a_1 \\ 
            a_2 \\
        \end{array} 
    } 
    \right]
\end{equation}

\section{Sistemas iterativos de transformações}

\subsection{Exemplos}

Dado um vetor inicial \( A \subset \mathbb{R}^2 \) e \( M \) contrações afins:
\[ 
    T_i: \mathbb{R}^2 \longrightarrow \mathbb{R}^2, \quad i = 1,2,...,M 
\]
Considere
\begin{equation}
    \mathlarger W(A) = \bigcup_{i = 1}^{M} T_i(A)
\end{equation}
e que
\begin{equation}
    W^n =
    \left\{
    	\begin{array}{l l}
    		 A                         & \mbox{se } x = 0 \\
    		 (W^{n} \circ W^{n-1})(A)  & \mbox{se } x > 0
    	\end{array}
    \right.
\end{equation}
Onde \( T_i(A) = \{ T_i(x) \mid x \in A \} \)
\begin{definition}

Um fractal gerado por IFS é um conjunto \( X \subset  \mathbb{R}^2 \) definido por 

\[ 
    \mathlarger{
        X = \lim_{x \rightarrow{\infty}} W^k(A)
    }  
\]

\end{definition}

\begin{theorem}\label{idenpotent}
Um fractal X do tipo IFS é o ponto fixo de uma contração W, ou seja,
\[ 
    W(X) = X 
\] 

\end{theorem}

\section{Execícios}

\begin{enumerate}
    \item Prove:
        \begin{enumerate}
            \item A proposição \ref{contract};
            \item O teorema \ref{idenpotent};
        \end{enumerate}
    
\end{enumerate}
