\documentclass[12pt]{report}
\usepackage{amsfonts}
\usepackage{amsthm}
\usepackage{amsmath}
\usepackage{array}
\usepackage{relsize}
\usepackage{graphicx}
\usepackage[thinlines]{easytable}
\usepackage[utf8]{inputenc}
\usepackage[portuguese]{babel}
\usepackage[final]{hyperref}

\newtheorem{theorem}{Teorema}
\newtheorem{corollary}{Corolário}[theorem]
\newtheorem{lemma}{Lema}
\theoremstyle{definition}
\newtheorem{definition}{Definição}
\newtheorem{proposition}{Proposição}
\newtheorem*{remark}{Observação}
\newtheorem{method}{Método}
\newtheorem{example}{Exemplo}


\title{Fractais, categorias e curvas algébricas}
\author{Kevyan Uehara de Moraes}
\date{Setembro 2019}


\begin{document}

\maketitle

\chapter*{Prefácio}\label{pre}

Este livro vem de anotações expandidas de aulas da matéria \textit{tópicos de matemática II}, ministrada no segundo semestre de 2019 pela professora Dra. Elizabeth Wegner Karas do Departamento de matemática da Universidade Federal do Paraná. Vem também dos conhecimentos adquiridos sobre curvas algérbicas e teoria de categorias no decorrer de minha iniciação cientifica orientado pelo professor Dr. Edson Ribeiro Alvares. 

Comecei a escrever este livro com o mesmo intuito da matéria: Um curso de fractais que não seja tão formal nem tão inicial e que não precise de muitos pré requisitos. Porém, encontrei-me ao mesmo encantado pela complexidade dos fractais. Junto com curvas algébricas e teoria de categorias, resolvi aumentar o recorte inicial do livro para uma apresentação categórica sobre, em especifico, curvas fractais.

Como todo livro da areá de exatas, existe um compromisso que deve ser feito entre didática, profundidade e formalismo. Didática é minha primeira preocupação. Formalismo e profundidade aparecem quando a intuição não é suficiente ou quando se entram necessários para o ensino de uma nova intuição. Alguns conceitos básicos serão relembrados, porém nem sempre com todo rigor.

Destino esse livro em especial a alunos do primeiro ano da graduação, ou que tenham conhecimentos de calculo e álgebra linear. Os primeiros capítulos são originários de anotações de aula e são destinados a quem procura um primeiro curso sobre fractais.

Os últimos capítulos decorrem de minha pesquisa junto ao professor Edson e são uma tentativa de explorar fractais com uma perspectiva diferente

De maneira geral, os exemplos contidos aqui explicam mais que as definições. É a partir deles que vai ocorrer os aprofundamentos e o verdadeiro ensino dos conceitos. 

\tableofcontents

\chapter{Introdução}
 
Fractais são figuras que tem certas características:
\begin{enumerate}

\item Estrutura fina
\item Auto-afinidade
\item Simplicidade na lei de formação
\item Difícil descrição analítica
    
\end{enumerate}
\chapter{Fractais gerados por processos geométricos}
\section{\textit{Cantor}}

{\renewcommand{\arraystretch}{1.5}
\begin{table}[h]
    \centering
        \begin{tabular}{| c | c | c | c |}
             \multicolumn{4}{c}{Conjunto de cantor} \\
             \hline
             Iteração & quantidade de pontos & total de pontos & comprimento total\\
             \hline
             0 & 1 & $ C_0 $  & $ C_0 $ \\
             \hline
             1 & 2 & $ \frac{ C_0 }{3} $ & $\frac{ 2C_{0} }{ 3 }$ \\
             \hline
             2 & 4 & $ \frac{ C_{0} }{9} $ & $\frac{ 4C_0}{9}$ \\
             \hline 
             3 & 8 & $( \frac{ C_{0} }{27} $ & $\frac{ 4C_0}{27}$ \\
             \hline
             ... & ... & ... & ... \\
             \hline
             k & $2^k$ & $ \frac{ C_{0} }{3^k} $ & $ C_0(\frac{2}{3})^3 $ \\
             \hline
        \end{tabular}
    \caption{Conjunto de cantor}
    \label{tab:cantor}
\end{table}}


e è facil mostrar que no infinto o comprimento total do conjunto é zero:
\[ C_{total} = \displaystyle \lim_{k\to\infty} C_0\left( \frac{2}{3} \right)^k = 0 \]


O conjunto de cantor tem certas características:
\begin{enumerate}

\item Compacto
\item Interior vazio
\item Não tem pontos isolados
\item Não enumerável
    
\end{enumerate}

\section{\textit{Koch}}


{\renewcommand{\arraystretch}{1.5}
\begin{table}[h]
    \centering
        \begin{tabular}{| c | c | c | c |}
             \multicolumn{4}{c}{Curva de Koch} \\
             \hline
             Iteração & quantidade de segmentos & total de segmentos & comprimento total\\
             \hline
             0 & 1 & $ C_0 $ & $ C_0 $ \\
             \hline
             1 & 4 & $ \frac{ C_0 }{3} $ & $\frac{ 4C_{0} }{ 3 }$) \\
             \hline
             2 & 16 & $ \frac{ C_{0} }{9} $ & $\frac{ 16C_0}{9}$ \\
             \hline 
             3 & 64 & $ \frac{ C_{0} }{27} $ & $\frac{ 64C_0}{27}$ \\
             \hline
             ... & ... & ... & ... \\
             \hline
             k & $3^k$ & $ \frac{ C_{0} }{3^k} $ & $ C_0(\frac{4}{3})^k $ \\
             \hline
        \end{tabular}
    \caption{Curva de koch}
    \label{tab:koch}
\end{table}}

e è fácil mostrar que no infinto o comprimento total do conjunto é \(\infty\):
\[ C_{total} = \displaystyle \lim_{k\to\infty} C_0\left( \frac{4}{3} \right)^k = \infty \]

e que o comprimento de um segmento é zero:
\[ C_{unidade} = \displaystyle \lim_{k\to\infty} C_0\left( \frac{1}{3} \right)^k = 0 \]

\section{\textit{Sierpinski}}

{\renewcommand{\arraystretch}{1.5}
\begin{table}[h]

    \centering
        \begin{tabular}{| c | c | c | c |}
             \multicolumn{4}{c}{Triangulo de Sierpinski} \\
             \hline
             Iteração & quantidade de triângulos & tamanho do segmento & comprimento total\\
             \hline
             0 & 1 & $ T_0 $ & $ T_0 $ \\
             \hline
             1 & 3 & $ \frac{ T_0 }{4} $ & $\frac{ 3T_{0} }{4}$ \\
             \hline
             2 & 9 & $ \frac{ T_{0} }{16} $ & $\frac{ 9T_0}{16}$ \\
             \hline 
             3 & 27 & $ \frac{ T_{0} }{64} $ & $\frac{ 27T_0}{64}$ \\
             \hline
             ... & ... & ... & ... \\
             \hline
             k & $3^k$ & $ \frac{ T_{0} }{4^k} $ & $ T_0(\frac{4}{3})^k $ \\
             \hline
        \end{tabular}
    \caption{Triangulo de Sierpinski}
    \label{tab:msier}
\end{table}}


\section{\textit{Peano}}

\section{Execícios}
\chapter{Conceito de dimensão}

Normalmente entendemos dimensão como sendo 

\begin{center}
    \item \( dimensão(Pontos) = 0 \)
    \item \( dimensão(Curvas) = 1 \)
    \item \( dimensão(Superfícies) = 2 \)
    \item \(\vdots \)
\end{center}    

Mas essas dimensão são dimensões \textit{topológicas}

\section{Dimensão Topológica X Dimensão espacial}


Dimensão espacial é dada por

\[ N = R^{-D} \]
\[ \log{N} = \log{R^{-D}} \]
\[ \log{N} = D\log{\frac{1}{R}} \]
\begin{equation}
     D = \frac{\log{N}}{\log{R^{-1}}}
\end{equation}

Onde

N = Numero de figuras,
R = Fator de redução,
D = Dimensão espacial

\begin{definition}[Fractal]
\label{frac}
Um fractal é uma figura onde a dimensão espacial é estritamente maior que a dimensão topológica
\end{definition}

\section{Exemplos}
\section{Contagem de caixas}

\[ D = \mathlarger{
        \lim_{
            l\to0
            } 
        \frac{
            \log{N}
        }{
            \log{
                \frac{
                    \delta
                    }{
                    l
                }
            }
        }
    }  \]

\begin{table}[h]
    \centering
        \begin{tabular}{| c | c | c | c |}
             \multicolumn{4}{c}{Samambaia} \\
             \hline
             l = r & $\delta/l$ & N & $\log{N}/\log{\frac{\delta}{l}}$ \\
             \hline
             128 & 8 & 27 & 1,5849 \\
             \hline
             64 & 16 & 90 & 1,6229 \\
             \hline
             32 & 32 & 315 & 1,6598 \\
             \hline 
        \end{tabular}
    \caption{Samambaia}
    \label{tab:sam}
\end{table}

\section{Execícios}
\chapter{Fractais gerados por IFS}
\section{Transformações lineares}
\subsection{Definição}

Seja \(U\) espaço vetorial sobre um corpo \( \mathbb{K} \), então
\[T:U \longrightarrow U\]
\\
é uma transformação linear se:

\begin{equation}
    T(\Vec{u} + \Vec{v}) = T(\Vec{u}) + T(\Vec{v})\ , \ \Vec{u}, \Vec{v} \in U
\end{equation}

\begin{equation}
    T(\alpha\Vec{u}) = \alpha T(\Vec{u}) \ , \ \alpha \in \mathbb{K}
\end{equation}

\begin{theorem}
se \(T\) é linear então \( T(0) = 0 \)
\end{theorem}

\begin{remark}
Neste capitulo trabalharemos em \( \mathbb{R}^{n} \), frequentemente em \( \mathbb{R}^{2} \) com a base canônica. T será sempre dado por:
\[ T: \mathbb{R}^n \longrightarrow \mathbb{R}^n \]
para algum \(n\)
\end{remark}

\subsection{Exemplos}

Tomemos T definido por

\[ T: \mathbb{R}^2 \longrightarrow \mathbb{R}^2 \]
\[ T(x, y) = (x, -y) =
  \left[ 
  {
    \begin{array}{c}
    x \\
   -y \\
    \end{array} 
    } 
  \right]
\]
note que aplicando as transformações nas bases canônicas obtemos
\[
T(1,0) = 
(1, 0) =
\left[ 
    {
        \begin{array}{c}
            1      \\
            0      \\
        \end{array} 
    } 
\right]
\]
\[
T(0,1) = 
(0, -1) =
\left[ 
    {
        \begin{array}{c}
            0      \\
            -1      \\
        \end{array} 
    } 
\right]
\]
E que 
\[ T(x, y) =
    \left[ 
    {
        \begin{array}{c c}
            1 &  0 \\
            0 & -1 \\
        \end{array} 
    } 
    \right]
    \left[ 
    {
        \begin{array}{c}
            x \\
            y \\
        \end{array} 
    } 
    \right]
    =
    \left[ 
    {
        \begin{array}{c}
            x \\
            -y \\
        \end{array} 
    } 
    \right]
\]
ou seja, \textbf{as imagens das transformações na base canônica de \( \mathbb{R}^{2} \) são precisamente as colunas da matriz que multiplica (x, y)}

\begin{remark}

mais genericamente, podemos descrever uma transformação \(T\) qualquer como


\[
    T: \mathbb{R}^{n} \longrightarrow \mathbb{R}^{n} 
\]
\[
    T(x) = A_{n \times n}x_{n \times 1} 
\]
Onde \(x\) é vetor do \(\mathbb{R}^{n}\) e \( A_{n \times n}\) é uma matriz \(n\) por \(n\).

Podemos definir \(A_{n x n}\) a partir das imagens de \( (1,0) \) e \( (0,1) \) como vetores coluna:
\[
    \left[ 
        {
            \begin{array}{c c}
                |\,    & |\,      \\
                T(1,0) & T(0,1)   \\
                |\,    & |\,      \\
            \end{array} 
        } 
    \right]
\]
\end{remark}

\section{Contração e Expansão}
\subsection{Definição}
\begin{definition}[contração]
    em geral, definimos que uma transformação \( T \) é uma contração quando:
        \[ 
        \exists \alpha \in [ \, 0,1\,) \, \, tal\, que
        \]
        \[
        \parallel T(u) - T(v) \parallel \: \leq \alpha \parallel u -v \parallel
        \]
    E definimos uma expansão de forma análoga, apenas invertendo o sinal da desigualdade.
    
\end{definition}

\subsection{exemplos}

\begin{proposition}\label{contract}
    Dado uma transformação T:
    \[
        T: \mathbb{R}^2 \longrightarrow \mathbb{R}^2
    \]
    \begin{equation}\label{escalar}
        T(x, y) = 
          \alpha
          \left[ 
            {
                \begin{array}{c c}
                    1 & 0 \\
                    0 & 1 \\
                \end{array} 
            } 
          \right]
          \left[ 
            {
                \begin{array}{c}
                    x \\
                    y \\
                \end{array} 
                } 
           \right]
           \, , \: \alpha \in \mathbb{R}
    \end{equation}
    se \( \alpha \in (0,1)\), então T é uma contração.
\end{proposition}
Neste caso, \(\alpha\) é o \textbf{Fator de contração} dessa transformação e ele que é usado no calculo da dimensão.
pode acontecer de queremos contrair os eixos de maneiras diferentes, como em
\[
E(x, y) = 
  \left[ 
    {
        \begin{array}{c c}
            \beta & 0      \\
            0     & \delta \\
        \end{array} 
    } 
  \right]
  \left[ 
    {
        \begin{array}{c}
            x \\
            y \\
        \end{array} 
        } 
   \right]
   ,
   \quad \beta,\delta \in [\,0, 1\,)
\]
Mas podemos visualizar \(E\) como:
\[
    E(x,y) = (\beta x, \delta y)
\]
Neste caso, o fator de contração é o \(min\{\beta, \delta\}\)
\section{Rotação}
\subsection{Definição}
seja

\[
    T: \mathbb{R}^2 \longrightarrow \mathbb{R}^2
\]
\[
    T(x) = R_{n \times n}x_{n \times 1}
\]
Usando o mesmo método de antes, podemos definir \(R\) a partir de suas imagens na base ortonormal

\begin{gather*}
    R(1,0) = R(\cos{\theta}, \sin{\theta})\\
    R(0,1) = R( - \sin{\theta}, \cos{\theta})
\end{gather*}
Então temos que
\begin{equation}
    R(x, y) = 
    \left[ 
    {
        \begin{array}{c c}
            \cos{\theta} & -\sin{\theta} \\
            \sin{\theta} &  \cos{\theta} \\
        \end{array} 
    } 
    \right]
    \left[ 
    {
        \begin{array}{c}
            x \\
            y \\
        \end{array} 
        } 
    \right]
\end{equation}

\subsection{exemplos}

\section{Transformações afins}

\subsection{Translação}
Seja
\[
    T: \mathbb{R}^2 \longrightarrow \mathbb{R}^2
\]
\begin{equation}
    T(x, y) = 
    \left[ 
    {
        \begin{array}{c}
            x \\
            y \\
        \end{array} 
        } 
    \right]
    +
    \left[ 
    {
        \begin{array}{c}
            a_1 \\ 
            a_2 \\
        \end{array} 
    } 
    \right]
    ,\quad a_1, a_2 \in \mathbb{R}
\end{equation}

\subsection{definição}

\begin{definition}

Uma \textbf{transformação afim} é uma transformação linear que admite translação e tem a forma:

\end{definition}

\begin{equation}
    A(x, y) = 
    \left[ 
    {
        \begin{array}{c c}
            \cos{\theta} & -\sin{\theta} \\
            \sin{\theta} &  \cos{\theta} \\
        \end{array} 
    } 
    \right]
    \left[ 
    {
        \begin{array}{c}
            \alpha  x \\
            \beta   y \\
        \end{array} 
        } 
    \right]
    +
    \left[ 
    {
        \begin{array}{c}
            a_1 \\ 
            a_2 \\
        \end{array} 
    } 
    \right]
\end{equation}
e se \( \beta = \delta\)

\begin{equation}
    A(x, y) =
    \alpha
    \left[ 
    {
        \begin{array}{c c}
            \cos{\theta} & -\sin{\theta} \\
            \sin{\theta} &  \cos{\theta} \\
        \end{array} 
    } 
    \right]
    \left[ 
    {
        \begin{array}{c}
            x \\
            y \\
        \end{array} 
        } 
    \right]
    +
    \left[ 
    {
        \begin{array}{c}
            a_1 \\ 
            a_2 \\
        \end{array} 
    } 
    \right]
\end{equation}

\section{Sistemas iterativos de transformações}

\subsection{Exemplos}

Dado um vetor inicial \( A \subset \mathbb{R}^2 \) e \( M \) contrações afins:
\[ 
    T_i: \mathbb{R}^2 \longrightarrow \mathbb{R}^2, \quad i = 1,2,...,M 
\]
Considere
\begin{equation}
    \mathlarger W(A) = \bigcup_{i = 1}^{M} T_i(A)
\end{equation}
e que
\begin{equation}
    W^n =
    \left\{
    	\begin{array}{l l}
    		 A                         & \mbox{se } x = 0 \\
    		 (W^{n} \circ W^{n-1})(A)  & \mbox{se } x > 0
    	\end{array}
    \right.
\end{equation}
Onde \( T_i(A) = \{ T_i(x) \mid x \in A \} \)
\begin{definition}

Um fractal gerado por IFS é um conjunto \( X \subset  \mathbb{R}^2 \) definido por 

\[ 
    \mathlarger{
        X = \lim_{x \rightarrow{\infty}} W^k(A)
    }  
\]

\end{definition}

\begin{theorem}\label{idenpotent}
Um fractal X do tipo IFS é o ponto fixo de uma contração W, ou seja,
\[ 
    W(X) = X 
\] 

\end{theorem}

\section{Execícios}

\begin{enumerate}
    \item Prove:
        \begin{enumerate}
            \item A proposição \ref{contract};
            \item O teorema \ref{idenpotent};
        \end{enumerate}
    
\end{enumerate}

\chapter{Sistemas Dinâmicos Complexos}

\section{Complexos}

Definimos o conjunto dos números complexos como:
\[
    \mathbb{C} = \{ x + iy\, |\, x,y \in \mathbb{R} \} 
\]
\[
    i = \sqrt{-1} \notin \mathbb{R}
\]

\subsection{Forma cartesiana}

escrevemos \(z\) como:
\[
    z = x + iy
\]
E tal como os números cartesianos, podemos descrever um numero complexo por uma dupla \((x,y)\).\\
O mesmo se aplica a calcular a distancia de um ponto a origem, ou seja, seu \textit{modulo}
\[
    r = |z| = \sqrt{x^2 + y^2}
\]
E em especial
\[
    |z^n| =  |z|^n
\]
\subsection{Forma polar}

Podemos descrever um ponto no plano por um angulo \(\theta\) e uma distancia \(r\) = \(|z|\). E damos \(z\) por um dupla \((r, \theta)\) 

\subsection{forma exponencial}

\begin{equation}
    \mathlarger{e^{i\theta} = \cos{\theta} + i\sin{\theta}}
\end{equation}
para radiciação,
\begin{align*}
    &z = re^{i \theta} \\
    &z^n = (re^{i \theta})^n\\
    &z^n = r^n e^{i n \theta} \\
\end{align*}


\section{Sistemas dinâmicos}
\subsection{Sistemas dinâmicos da função \(z^2\)}

Considere a função:

\[ 
    f: \mathbb{C} \longrightarrow \mathbb{C} 
\]
\[ 
    f(z) := z^2
\]
Dado \(z_ 0\) e a sequencia \(\{z_k\}\) definida por:
\[
    z_k = f(z_{k-1})
\]
\begin{align*}
    z_0 &\\
    z_1 &= f(z_0) = f^2(z_0) = (z_0)^2 \\
    z_2 &= f(z_1) = f^3(z_0) = (z_0)^4 \\
    \vdots \\
    z_k &= f(z_{k-1}) = f^k(z_0) = (z_0)^{2^k} 
\end{align*}

\begin{definition}[Orbita de \(z_0\)]
    Dizemos que a Orbita de \(z_0\) para alguma função \(f\) é o conjunto \(\mathcal{O}(z_0)\) definido por
    \[
        \mathcal{O}(z_0) := \{z_0, z_1, ..., z_k, ...\}
    \]
\end{definition}
\begin{definition}[ponto fixo de \(f(z)\)]
    Dizemos que algum \( z \) é ponto fixo de \(f(z)\) quando
    \[ 
        f(z) = z 
    \]
\end{definition}
Vemos por exemplo que
\[
    \mathcal{O}(1) = {1}
\]
ou seja, 1 é ponto fixo de \( f(z) = z^2 \)
\[
    \mathcal{O}(-1) = \{-1, 1\}
\]
\[
    \mathcal{O}\left(\frac{1}{\sqrt{2}} (1 + i)\right) = \left\{ \frac{1}{\sqrt{2}} (1 + i), i , -1 , 1 \right\}
\]
\[
    r = \sqrt{\frac{1}{2}} = \frac{1}{\sqrt{2}} \leq 1
\]
Da forma análoga a qual definimos a orbita de um numero \(z_0\), podemos definir uma \textit{Orbita regressiva}, que denotaremos por \(\mathcal{O}_{-}\).
\begin{definition}[Orbita regressiva de \(z_0\)]
    A orbita regressiva é a sequencia {\(z_{-k}\)} dada por:
    \[
        f(z_{-k}) = z_{-k + 1}
    \]
\end{definition}

\section{Execícios}

\chapter{Julia, Mandebrot e seus Conjuntos}

\section{Exercícios}

\begin{enumerate}
    \item Prove:
        \begin{enumerate}
            \item 
                Se \(z_0\) é repulsor, então \(z_0 \in \mathbf{J}\)
            \item
                Se \(x_i \in \mathcal{O}(x_i)\), então, \(\lambda_{x_i} = \lambda_{x_j}, \forall i,j\)
            \item
                Existe apenas uma orbita \( \mathcal{O}(z)\), tal que \(\lambda_{z}\) não é repulsora, para \(f(z) = z^2 + c \)
            \item 
                Para \( c = -2 \in \mathbf{M} \)
            \item 
                \(f_c \) tem orbita atratora ou superatratora \( \Longleftrightarrow |c-1| < \frac{1}{4}\)
            \item
                \(f_c \) tem orbita indiferente de periodo 2 \( \Longleftrightarrow |c-1| = \frac{1}{4}\)
            \item
                Se \(z\) é ponto periódico de período \(n\) de \(f_c\), então \(z\) é ponto periódico de período \(nm, \forall m \in \mathbb{N} \)
                \\
                ou seja,
                \\
                \[ 
                    f_c^n(z) = z \Longrightarrow f_c^{nm}(z) = z, \forall m \in \mathbb{N} 
                \]
        \end{enumerate}
    \item
        Mostre que se \( f_c^n(z) = z \) e \(f_c(z) = z \), então 
        \(f_c(z) - z \) é um fator de \( f_c^n(z) - \overline{z} \)
        
        
\end{enumerate}

\chapter{Um pouco de teoria de categorias}

\chapter{Fractais categóricos}

\begin{appendix}
  \listoffigures
  \listoftables
\end{appendix}

\end{document}
