\chapter{Fractais gerados por processos geométricos}
\section{\textit{Cantor}}

{\renewcommand{\arraystretch}{1.5}
\begin{table}[h]
    \centering
        \begin{tabular}{| c | c | c | c |}
             \multicolumn{4}{c}{Conjunto de cantor} \\
             \hline
             Iteração & quantidade de pontos & total de pontos & comprimento total\\
             \hline
             0 & 1 & $ C_0 $  & $ C_0 $ \\
             \hline
             1 & 2 & $ \frac{ C_0 }{3} $ & $\frac{ 2C_{0} }{ 3 }$ \\
             \hline
             2 & 4 & $ \frac{ C_{0} }{9} $ & $\frac{ 4C_0}{9}$ \\
             \hline 
             3 & 8 & $( \frac{ C_{0} }{27} $ & $\frac{ 4C_0}{27}$ \\
             \hline
             ... & ... & ... & ... \\
             \hline
             k & $2^k$ & $ \frac{ C_{0} }{3^k} $ & $ C_0(\frac{2}{3})^3 $ \\
             \hline
        \end{tabular}
    \caption{Conjunto de cantor}
    \label{tab:cantor}
\end{table}}


e è facil mostrar que no infinto o comprimento total do conjunto é zero:
\[ C_{total} = \displaystyle \lim_{k\to\infty} C_0\left( \frac{2}{3} \right)^k = 0 \]


O conjunto de cantor tem certas características:
\begin{enumerate}

\item Compacto
\item Interior vazio
\item Não tem pontos isolados
\item Não enumerável
    
\end{enumerate}

\section{\textit{Koch}}


{\renewcommand{\arraystretch}{1.5}
\begin{table}[h]
    \centering
        \begin{tabular}{| c | c | c | c |}
             \multicolumn{4}{c}{Curva de Koch} \\
             \hline
             Iteração & quantidade de segmentos & total de segmentos & comprimento total\\
             \hline
             0 & 1 & $ C_0 $ & $ C_0 $ \\
             \hline
             1 & 4 & $ \frac{ C_0 }{3} $ & $\frac{ 4C_{0} }{ 3 }$) \\
             \hline
             2 & 16 & $ \frac{ C_{0} }{9} $ & $\frac{ 16C_0}{9}$ \\
             \hline 
             3 & 64 & $ \frac{ C_{0} }{27} $ & $\frac{ 64C_0}{27}$ \\
             \hline
             ... & ... & ... & ... \\
             \hline
             k & $3^k$ & $ \frac{ C_{0} }{3^k} $ & $ C_0(\frac{4}{3})^k $ \\
             \hline
        \end{tabular}
    \caption{Curva de koch}
    \label{tab:koch}
\end{table}}

e è fácil mostrar que no infinto o comprimento total do conjunto é \(\infty\):
\[ C_{total} = \displaystyle \lim_{k\to\infty} C_0\left( \frac{4}{3} \right)^k = \infty \]

e que o comprimento de um segmento é zero:
\[ C_{unidade} = \displaystyle \lim_{k\to\infty} C_0\left( \frac{1}{3} \right)^k = 0 \]

\section{\textit{Sierpinski}}

{\renewcommand{\arraystretch}{1.5}
\begin{table}[h]

    \centering
        \begin{tabular}{| c | c | c | c |}
             \multicolumn{4}{c}{Triangulo de Sierpinski} \\
             \hline
             Iteração & quantidade de triângulos & tamanho do segmento & comprimento total\\
             \hline
             0 & 1 & $ T_0 $ & $ T_0 $ \\
             \hline
             1 & 3 & $ \frac{ T_0 }{4} $ & $\frac{ 3T_{0} }{4}$ \\
             \hline
             2 & 9 & $ \frac{ T_{0} }{16} $ & $\frac{ 9T_0}{16}$ \\
             \hline 
             3 & 27 & $ \frac{ T_{0} }{64} $ & $\frac{ 27T_0}{64}$ \\
             \hline
             ... & ... & ... & ... \\
             \hline
             k & $3^k$ & $ \frac{ T_{0} }{4^k} $ & $ T_0(\frac{4}{3})^k $ \\
             \hline
        \end{tabular}
    \caption{Triangulo de Sierpinski}
    \label{tab:msier}
\end{table}}


\section{\textit{Peano}}

\section{Execícios}