\chapter{Sistemas Dinâmicos Complexos}

\section{Complexos}

Definimos o conjunto dos números complexos como:
\[
    \mathbb{C} = \{ x + iy\, |\, x,y \in \mathbb{R} \} 
\]
\[
    i = \sqrt{-1} \notin \mathbb{R}
\]

\subsection{Forma cartesiana}

escrevemos \(z\) como:
\[
    z = x + iy
\]
E tal como os números cartesianos, podemos descrever um numero complexo por uma dupla \((x,y)\).\\
O mesmo se aplica a calcular a distancia de um ponto a origem, ou seja, seu \textit{modulo}
\[
    r = |z| = \sqrt{x^2 + y^2}
\]
E em especial
\[
    |z^n| =  |z|^n
\]
\subsection{Forma polar}

Podemos descrever um ponto no plano por um angulo \(\theta\) e uma distancia \(r\) = \(|z|\). E damos \(z\) por um dupla \((r, \theta)\) 

\subsection{forma exponencial}

\begin{equation}
    \mathlarger{e^{i\theta} = \cos{\theta} + i\sin{\theta}}
\end{equation}
para radiciação,
\begin{align*}
    &z = re^{i \theta} \\
    &z^n = (re^{i \theta})^n\\
    &z^n = r^n e^{i n \theta} \\
\end{align*}


\section{Sistemas dinâmicos}
\subsection{Sistemas dinâmicos da função \(z^2\)}

Considere a função:

\[ 
    f: \mathbb{C} \longrightarrow \mathbb{C} 
\]
\[ 
    f(z) := z^2
\]
Dado \(z_ 0\) e a sequencia \(\{z_k\}\) definida por:
\[
    z_k = f(z_{k-1})
\]
\begin{align*}
    z_0 &\\
    z_1 &= f(z_0) = f^2(z_0) = (z_0)^2 \\
    z_2 &= f(z_1) = f^3(z_0) = (z_0)^4 \\
    \vdots \\
    z_k &= f(z_{k-1}) = f^k(z_0) = (z_0)^{2^k} 
\end{align*}

\begin{definition}[Orbita de \(z_0\)]
    Dizemos que a Orbita de \(z_0\) para alguma função \(f\) é o conjunto \(\mathcal{O}(z_0)\) definido por
    \[
        \mathcal{O}(z_0) := \{z_0, z_1, ..., z_k, ...\}
    \]
\end{definition}
\begin{definition}[ponto fixo de \(f(z)\)]
    Dizemos que algum \( z \) é ponto fixo de \(f(z)\) quando
    \[ 
        f(z) = z 
    \]
\end{definition}
Vemos por exemplo que
\[
    \mathcal{O}(1) = {1}
\]
ou seja, 1 é ponto fixo de \( f(z) = z^2 \)
\[
    \mathcal{O}(-1) = \{-1, 1\}
\]
\[
    \mathcal{O}\left(\frac{1}{\sqrt{2}} (1 + i)\right) = \left\{ \frac{1}{\sqrt{2}} (1 + i), i , -1 , 1 \right\}
\]
\[
    r = \sqrt{\frac{1}{2}} = \frac{1}{\sqrt{2}} \leq 1
\]
Da forma análoga a qual definimos a orbita de um numero \(z_0\), podemos definir uma \textit{Orbita regressiva}, que denotaremos por \(\mathcal{O}_{-}\).
\begin{definition}[Orbita regressiva de \(z_0\)]
    A orbita regressiva é a sequencia {\(z_{-k}\)} dada por:
    \[
        f(z_{-k}) = z_{-k + 1}
    \]
\end{definition}

\section{Execícios}

\chapter{Julia, Mandebrot e seus Conjuntos}

\section{Exercícios}

\begin{enumerate}
    \item Prove:
        \begin{enumerate}
            \item 
                Se \(z_0\) é repulsor, então \(z_0 \in \mathbf{J}\)
            \item
                Se \(x_i \in \mathcal{O}(x_i)\), então, \(\lambda_{x_i} = \lambda_{x_j}, \forall i,j\)
            \item
                Existe apenas uma orbita \( \mathcal{O}(z)\), tal que \(\lambda_{z}\) não é repulsora, para \(f(z) = z^2 + c \)
            \item 
                Para \( c = -2 \in \mathbf{M} \)
            \item 
                \(f_c \) tem orbita atratora ou superatratora \( \Longleftrightarrow |c-1| < \frac{1}{4}\)
            \item
                \(f_c \) tem orbita indiferente de periodo 2 \( \Longleftrightarrow |c-1| = \frac{1}{4}\)
            \item
                Se \(z\) é ponto periódico de período \(n\) de \(f_c\), então \(z\) é ponto periódico de período \(nm, \forall m \in \mathbb{N} \)
                \\
                ou seja,
                \\
                \[ 
                    f_c^n(z) = z \Longrightarrow f_c^{nm}(z) = z, \forall m \in \mathbb{N} 
                \]
        \end{enumerate}
    \item
        Mostre que se \( f_c^n(z) = z \) e \(f_c(z) = z \), então 
        \(f_c(z) - z \) é um fator de \( f_c^n(z) - \overline{z} \)
        
        
\end{enumerate}