\chapter{Conceito de dimensão}

Normalmente entendemos dimensão como sendo 

\begin{center}
    \item \( dimensão(Pontos) = 0 \)
    \item \( dimensão(Curvas) = 1 \)
    \item \( dimensão(Superfícies) = 2 \)
    \item \(\vdots \)
\end{center}    

Mas essas dimensão são dimensões \textit{topológicas}

\section{Dimensão Topológica X Dimensão espacial}


Dimensão espacial é dada por

\[ N = R^{-D} \]
\[ \log{N} = \log{R^{-D}} \]
\[ \log{N} = D\log{\frac{1}{R}} \]
\begin{equation}
     D = \frac{\log{N}}{\log{R^{-1}}}
\end{equation}

Onde

N = Numero de figuras,
R = Fator de redução,
D = Dimensão espacial

\begin{definition}[Fractal]
\label{frac}
Um fractal é uma figura onde a dimensão espacial é estritamente maior que a dimensão topológica
\end{definition}

\section{Exemplos}
\section{Contagem de caixas}

\[ D = \mathlarger{
        \lim_{
            l\to0
            } 
        \frac{
            \log{N}
        }{
            \log{
                \frac{
                    \delta
                    }{
                    l
                }
            }
        }
    }  \]

\begin{table}[h]
    \centering
        \begin{tabular}{| c | c | c | c |}
             \multicolumn{4}{c}{Samambaia} \\
             \hline
             l = r & $\delta/l$ & N & $\log{N}/\log{\frac{\delta}{l}}$ \\
             \hline
             128 & 8 & 27 & 1,5849 \\
             \hline
             64 & 16 & 90 & 1,6229 \\
             \hline
             32 & 32 & 315 & 1,6598 \\
             \hline 
        \end{tabular}
    \caption{Samambaia}
    \label{tab:sam}
\end{table}

\section{Execícios}