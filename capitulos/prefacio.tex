\chapter*{Prefácio}\label{pre}

Este livro vem de anotações expandidas de aulas da matéria \textit{tópicos de matemática II}, ministrada no segundo semestre de 2019 pela professora Dra. Elizabeth Wegner Karas do Departamento de matemática da Universidade Federal do Paraná. Vem também dos conhecimentos adquiridos sobre curvas algérbicas e teoria de categorias no decorrer de minha iniciação cientifica orientado pelo professor Dr. Edson Ribeiro Alvares. 

Comecei a escrever este livro com o mesmo intuito da matéria: Um curso de fractais que não seja tão formal nem tão inicial e que não precise de muitos pré requisitos. Porém, encontrei-me ao mesmo encantado pela complexidade dos fractais. Junto com curvas algébricas e teoria de categorias, resolvi aumentar o recorte inicial do livro para uma apresentação categórica sobre, em especifico, curvas fractais.

Como todo livro da areá de exatas, existe um compromisso que deve ser feito entre didática, profundidade e formalismo. Didática é minha primeira preocupação. Formalismo e profundidade aparecem quando a intuição não é suficiente ou quando se entram necessários para o ensino de uma nova intuição. Alguns conceitos básicos serão relembrados, porém nem sempre com todo rigor.

Destino esse livro em especial a alunos do primeiro ano da graduação, ou que tenham conhecimentos de calculo e álgebra linear. Os primeiros capítulos são originários de anotações de aula e são destinados a quem procura um primeiro curso sobre fractais.

Os últimos capítulos decorrem de minha pesquisa junto ao professor Edson e são uma tentativa de explorar fractais com uma perspectiva diferente

De maneira geral, os exemplos contidos aqui explicam mais que as definições. É a partir deles que vai ocorrer os aprofundamentos e o verdadeiro ensino dos conceitos. 